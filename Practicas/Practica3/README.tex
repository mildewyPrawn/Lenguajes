% Lenguajes de Programación 2019-1
% Plantilla para reportes de laboratorio.

\documentclass{article}

\usepackage{authblk}
\usepackage[utf8]{inputenc}
\usepackage{listings}
{\lstset{}%
  \csname lst@SetFirstLabel\endcsname}
{\csname lst@SaveFirstLabel\endcsname}
\lstset{
  basicstyle=\small\ttfamily,
  flexiblecolumns=false,
  basewidth={0.5em,0.45em},
  literate={+}{{$+$}}1 {/}{{$/$}}1 {*}{{$*$}}1 {=}{{$=$}}1
  {>}{{$>$}}1 {<}{{$<$}}1 {\\}{{$\lambda$}}1
  {\\\\}{{\char`\\\char`\\}}1
  {->}{{$\rightarrow$}}2 {>=}{{$\geq$}}2 {<-}{{$\leftarrow$}}2
  {<=}{{$\leq$}}2 {=>}{{$\Rightarrow$}}2 
  {\ .}{{$\circ$}}2 {\ .\ }{{$\circ$}}2
  {>>}{{>>}}2 {>>=}{{>>=}}2
  {|}{{$\mid$}}1               
}

\title{Práctica 3}
\author{314032324	Galeana Araujo, Emiliano\\
	314011163	Miranda Sánchez, Kevin Ricardo}
\affil{Facultad de Ciencias, UNAM}
\date{Fecha de entrega: Miercoles 3 de Octubre 2018}

\begin{document}

\maketitle

\section{Descripción del programa}
Expresiones del Calculo Lambda, el cálculo lambda consiste siplemente en tres
términos y todas las combinaciones recursivas válidas de estos términos. Las
definimos como sigue.
\begin{lstlisting}[language=Haskell]
-- | Identifier. Tipo que define un nombre de variables como 
-- |             una cadena de texto.
type Identifier = String

-- | Expr. Tipo que representa una expresion lambda sin tipos.
data Expr = Var Identifier
          | Lam Identifier Expr
          | App Expr Expr deriving(Eq)
           
-- | Substitution. Tipo que representa la sustitucion.
type Substitution = ( Identifier , Expr )
\end{lstlisting}
Por ejemplo, la siguiente expresión $\lambda x. \lambda y. xy$, la
representaremos en Haskell como sigue. $\setminus x \rightarrow \setminus y
\rightarrow (x y)$. Todo esto representa las expresiones del calculo lambda sin
tipos y el reducto para la funcion de sustitucion.\\
\\Se realizaron las siguiente funciones que representan la semántica operacional
en en calculo lambda sin tipos.

\begin{lstlisting}[language=Haskell]
-- | frVars. Obtiene el conjunto de variables libres de una expresion.
frVars :: Expr -> [Identifier]

-- | lkVars. Obtiene el conjunto de variables ligadas de una expresion.
lkVars :: Expr -> [Identifier]

-- | incrVar. Dado un identificador, si este no termina en numero
-- |          le agrega el sufijo 1, en caso contrario toma el valor del numero
-- |          y lo incrementa en 1.
incrVar :: Identifier -> Identifier


-- | alphaExpr. Toma una expresion lambda y devuelve una alpha-equivalente 
-- |            utilizando la funcion incrVar hasta encontrar un nombre que no
-- |            aparezca en el cuerpo.
alphaExpr :: Expr -> Expr

-- | subst. Aplica la sustitucion a la expresion dada.
subst :: Expr -> Substitution -> Expr

-- | beta. Aplica un paso de la beta reduccion.
beta :: Expr -> Expr

-- | locked. Determina si una expresion esta bloqueada, es decir, no se pueden
-- |         hacer mas reducciones.
locked :: Expr -> Bool

-- | eval. Evalua una expresion lambda aplicando beta reducciones hasta quedar
-- |       bloqueada.
eval :: Expr -> Expr

\end{lstlisting}

\section{Entrada y ejecución}

El programa es interpretado por \texttt{GCHI} de la siguiente forma
\begin{lstlisting}[]
  ~:ghci Practica3.hs
\end{lstlisting}

En el programa puede probar algunos ejemplos de ejecucion, esciribiendo
simplemente  el nombre del ejemplo que se quiere ejecutar.\\

\subsection{Calculo lambda}
En el programa se encuentran las lineas de codigo.

\begin{lstlisting}[language=Haskell]
---FRVARS-----------
ejemplo = frVars (App (Lam "x" (App ( Var "x" ) ( Var "y" ) ) )(Lam "z" ( Var "z" ) ) )
ejemplo2 = frVars (Lam "f" (App (App (Var "f")(Lam "x"(App(App(Var "f")(Var
"x"))(Var "x" ))))(Lam "x"(App(App(Var "f")(Var "x" ))( Var "x")))))
---LKVARS---------
ejemplo3 = lkVars (App (Lam "x" (App ( Var "x" ) ( Var "y" ))) (Lam "z" ( Var
"z" )))
ejemplo4 = lkVars (Lam "f"(App(App(Var "f" )(Lam "x"(App(App(Var "f")(Var "x" ))
(Var "x" ))))(Lam "x" (App(App(Var "f" )(Var "x"))(Var "x" )))))
----INCRVAR----
ejemplo5 = incrVar "elem"
ejemplo6 = incrVar "x97"

-----ALPHAEXPR-------------------------
ejemplo7 = alphaExpr (Lam "x" (Lam "y" (App (Var "x" ) (Var "y" ))))
ejemplo8 = alphaExpr (Lam "x" (Lam "x1" (App ( Var "x" ) ( Var "x1"))))

--------SUBST------
ejemplo9 = subst (Lam "x" (App ( Var "x" ) ( Var "y" ) ) ) ( "y" , Lam "z" ( Var
"z" ))
ejemplo10 = subst (Lam "x" ( Var "y" )) ( "y" , Var "x" )

-----BETA-----
ejemplo11 = beta (App (Lam "x" (App ( Var "x" ) ( Var "y" ))) (Lam "z" ( Var "z"
)))
ejemplo20 = beta (App (Lam "n" (Lam "s" (Lam "z" (App ( Var "s" ) (App (App (
Var "n" ) ( Var "s" ) ) ( Var "z" ) ) ) ) ) ) (Lam "s" (Lam "z" ( Var "z" ) ) ) )
-----------LOCKED--------
ejemplo12 = locked (Lam "s" (Lam "z" ( Var "z" ) ) )
ejemplo13 = locked (Lam "x" (App (Lam "x" ( Var "x" ))( Var "z" )))

ejemplo14 = eval (App (Lam "n" (Lam "s" (Lam "z" (App ( Var "s" ) (App (App (
Var "n" )  ( Var "s" ) ) ( Var "z" ) ) ) ) ) ) (Lam "s" (Lam "z" ( Var "z" ) ) ) )
ejemplo15 = eval (App (Lam "n"(Lam "s"(Lam "z" (App (Var "s")(App (App (Var "n")
(Var "s"))(Var "z" ))))))(Lam "s" (Lam "z" (App ( Var "s" )(Var "z" )))))

cero = Lam "s" (Lam "z" (Var "z"))
uno = Lam "s1" (Lam "z1" (App (Var "s1") (Var "z1")))
suc = Lam "n" (Lam "s2" (Lam "z2" (App (Var "s2") (App (App (Var "n") (Var
"s2"))  (Var "z2")))))

ejemplo16 = eval (App suc cero)
ejemplo17 = eval (App suc uno)
\end{lstlisting}
Entonces, para ejecutar algun de los ejemplos, basta escribir el nombre de la siguiente manera:
\begin{lstlisting}[language=Haskell]
*Practica3> ejemplo15
\s ->\z -> (s (s z))
*Practica3> ejemplo12
True
*Practica3> ejemplo3
["x","z"]
*Practica3> ejemplo9
\x -> (x \z -> z)
\end{lstlisting}
Nota, el regreso no está como en nuestra implementación, ya que tiene el simbolo
$\lambda$, pero se puede ver que ambas representaciones son equivalentes.
\section{Conclusiones}
La parte de \texttt{locked} fue la más complicada, ya que hicimos varias
implementaciones y con los ejemplos del PDF estaban bien, pero cuendo metíamos
otros ejemplos, no salía lo que esperábamos, y en algún momento pensamos en usar
\texttt{locked} para hacer \texttt{eval}, al final no lo necesitamos y pudimos
arreglar \texttt{locked} en su mayoría.
%Bibliografia

\begin{thebibliography}{9}

\bibitem{lamport94}
  Archivero, curso de Lenguajes de Programación 2019-1

\end{thebibliography}

\end{document}