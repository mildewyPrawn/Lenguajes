% Lenguajes de Programación 2019-1
% Plantilla para reportes de laboratorio.

\documentclass{article}

\usepackage{authblk}
\usepackage[utf8]{inputenc}
\usepackage{listings}
\usepackage{proof}

\title{Práctica 2}
\author{Galeana Araujo, Emiliano, 314032324\\
	Miranda Sánchez Kevin, 314011163}
\affil{Facultad de Ciencias, UNAM}
\date{Fecha de entrega:19 Septiembre 2018 }

\begin{document}

\maketitle
\section{Semántica Dinámica.\\}

\setlength{\inferLineSkip}{6pt}

\[\infer{bool[true] \hspace{1em} valor}{}\ \hspace{1em} \infer{bool[false] \hspace{1em} valor}{}\] 
\infer[(eandf)]{and(bool[n],bool[m]) \rightarrow and(n\&\&m)}{} \hspace{1em}
\infer[(eandd)]{and(bool[n], t_{2}) \rightarrow and(bool[n], t_{2}')}{ t_{2} \rightarrow t_{2}'}\\
\\\\\infer[(eandi)]{and(t_{1}, t_{2}) \rightarrow and(t_{1}' , t_{2}')}{ t_{1} \rightarrow t_{1}'}\\
\\\\\infer[(eorf)]{or(bool[n],bool[m]) \rightarrow or(n||m)}{} \hspace{1em}
\infer[(eord)]{or(bool[n], t_{2}) \rightarrow or(bool[n], t_{2}')}{ t_{2} \rightarrow t_{2}'}\\
\\\\\infer[(eori)]{or(t_{1}, t_{2}) \rightarrow or(t_{1}' , t_{2})}{ t_{1} \rightarrow t_{1}'}\\
\\\\\infer[(enotn)]{not(bool[n]) \rightarrow \textbf{not}(n)}{}\\\\\\\infer[(enotn)]{not(t) \rightarrow not(t')}{t \rightarrow t'}\\
\infer[(elt0)]{lt(num[n],num[n]) \rightarrow bool[false]}{} \hspace{1em}
\infer[(eltf)]{lt(num[n],num[m]) \rightarrow bool[n<m]}{}
\\\\\infer[(elti)]{lt(t_{1}, t_{2}) \rightarrow lt( t_{1}',t_{2})}{ t_{1}  \rightarrow t_{1}'} \hspace{1em}\
\infer[(eltd)]{lt(num[n],t_{2}) \rightarrow lt[num[n], t_{2}']}{t_{2} \rightarrow t_{2}'} \\
\\\\\infer[(egtf0]{gt(num[n],num[n]) \rightarrow bool[false]}{} \hspace{1em}
\infer[(egtf]{gt(num[n],num[m]) \rightarrow bool[n<m]}{} \\
\\\\\infer[(egti)]{gt(t_{1}, t_{2}) \rightarrow gt( t_{1}',t_{2})}{ t_{1}  \rightarrow t_{1}'} \hspace{1em} \hspace{1em}
\infer[(egtd)]{gt(num[n],t_{2}) \rightarrow gt[num[n], t_{2}']}{t_{2} \rightarrow t_{2}'} \\
\\\\\infer[(eeq0)]{eq(num[n],num[n]) \rightarrow bool[true]}{} \hspace{1em}
\infer[(eeqf)]{eq(num[n],num[m]) \rightarrow bool[n==m]}{} \\
\\\\\infer[(eeqi)]{eq(t_{1}, t_{2}) \rightarrow eq( t_{1}',t_{2})}{ t_{1}  \rightarrow t_{1}'} \hspace{1em}
\infer[(eeqd)]{eq(num[n],t_{2}) \rightarrow eq[num[n], t_{2}']}{t_{2} \rightarrow t_{2}'} \\

\section{Semántica Estática}
$$
\infer[\textbf{(tand)}]            
{\Gamma \vdash and(t_1,t_2): Bool }
{
	\Gamma \vdash t_1 : Bool \quad \Gamma \vdash t_2 : Bool
}
\quad
\infer[\textbf{(tor)}]
{\Gamma \vdash or(t_1,t_2): Bool }
{
	\Gamma \vdash t_1 : Bool \quad \Gamma \vdash t_2 : Bool
}
$$

$$
\infer[\textbf{(tneg)}]            
{\Gamma \vdash neg(t): Bool }
{
	\Gamma \vdash t : Bool
}
\quad
\infer[\textbf{(tlt)}]
{\Gamma \vdash lt(t_1,t_2): Bool }
{
	\Gamma \vdash t_1 : Nat \quad \Gamma \vdash t_2 : Nat
}
$$

$$
\infer[\textbf{(tgt)}]            
{\Gamma \vdash gt(t_1,t_2): Bool }
{
	\Gamma \vdash t_1 : Nat \quad \Gamma \vdash t_2 : Nat
}
\quad
\infer[\textbf{(teq)}]
{\Gamma \vdash eq(t_1,t_2): Bool }
{
	\Gamma \vdash t_1 : Nat  \quad \Gamma \vdash t_2 : Nat
}
$$
\section{Descripción del programa}

En esta practica podemos poner en practica el tema de semántica visto en clase.
Se usara la semántica operacional para definir el comportamiento de los programas a través de un sistema de transiciones.

Las definiciones a definir son las siguientes:

\begin{itemize}
	\item eval1. Función que devuelve la transición tal que eval1 e = e' syss e -$>$ e'.
				eval1 :: Exp -$>$ Exp
	 \item evals. Funcion que devuelve la transicion tal que evals e = e' syss e -$>$* e' y e' esta bloqueado.\\
	 evals :: Exp -$>$ Exp
	 \item eval. Funcion que devuelve la evaluación de un programa tal que eval e=e' syss e -$>$* e' y e' es un valor. En caso de que e' no sea un valor deberá mostrar un mensaje de error particular del operador que lo causó.\\
	 eval :: Exp -$>$ Exp
	 \item vt. Funcion que verifica el tipado de un programa tal que vt $\Gamma$ e T = True syss $\Gamma \vdash$ e:T
	 vt :: TypCtxt -$>$ Exp -$>$ Type -$>$ Bool
\end{itemize}

\section{Entrada y ejecución}
El programa es interpretado por GCHI de la siguiente forma
 ~:ghci Practica2.\\
\\\\Algunos ejemplos:\\
 
\begin{lstlisting}[language=Haskell]
Pratica2> eval1 (Add ( I 1 ) ( I 2 ) )
I 3
Practica2> evals ( Let "x" (Add ( I 1 ) ( I 2 ) ) (Eq (V "x" ) ( I 0 ) ) )
B False
Practica2> eval(Or (Eq (Add ( I 0 ) ( I 0 ) ) ( I 0 ) ) (Eq ( I 1 ) ( I 1 0 ) ) )
B True
Practica2> vt [ ( "x" , Boolean ) ] ( I f (B True) (B False ) ( Var "x" ) ) Boolean
True
\end{lstlisting}

\section{Conclusiones}
Esta ha sido la practica mas difícil hasta ahora y aún si fue emocionante hacerla. 
Parece que logramos entender como funciona la semántica estática y la dinámica  

%Bibliografia

\begin{thebibliography}{9}

\bibitem{lamport94}
  Leslie Lamport,
  \emph{\LaTeX: a document preparation system},
  Addison Wesley, Massachusetts,
  2nd edition,
  1994.

\end{thebibliography}

\end{document}