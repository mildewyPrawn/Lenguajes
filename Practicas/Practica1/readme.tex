% Lenguajes de Programación 2019-1
% Plantilla para reportes de laboratorio.

\documentclass[spanish,12pt,letterpaper]{article}

\usepackage[spanish]{babel}
\usepackage[utf8]{inputenc}
\usepackage{authblk}
\usepackage{listings}
\usepackage{dsfont}
\lstnewenvironment{code}
{\lstset{}%
  \csname lst@SetFirstLabel\endcsname}
{\csname lst@SaveFirstLabel\endcsname}
\lstset{
  basicstyle=\small\ttfamily,
  flexiblecolumns=false,
  basewidth={0.5em,0.45em},
  literate={+}{{$+$}}1 {/}{{$/$}}1 {*}{{$*$}}1 {=}{{$=$}}1
  {>}{{$>$}}1 {<}{{$<$}}1 {\\}{{$\lambda$}}1
  {\\\\}{{\char`\\\char`\\}}1
  {->}{{$\rightarrow$}}2 {>=}{{$\geq$}}2 {<-}{{$\leftarrow$}}2
  {<=}{{$\leq$}}2 {=>}{{$\Rightarrow$}}2 
  {\ .}{{$\circ$}}2 {\ .\ }{{$\circ$}}2
  {>>}{{>>}}2 {>>=}{{>>=}}2
  {|}{{$\mid$}}1               
}

\title{Práctica 01}
\author{Galeana Araujo Emiliano\\
Miranda Sánchez Kevin Ricardo}
\affil{Facultad de Ciencias, UNAM}
\date{Fecha de entrega: 26 de Agosto de 2018}

\begin{document}

\maketitle

\section{Descripción del programa}
En este semanal vemos una implementación de \textit{Postfix}, que es una
secuencia parentizada que consiste en una palabra reservada \texttt{Postfix},
seguida de un número natural que indica el número de argumentos que recibe el
programa, seguido de cero o más comandos.
\subsection{Postfix}

\begin{code}
  data Command = I Int | ADD | DIV | Eq | EXEC | Gt | Lt 
                 | MUL | NGET | POP | REM | SEL | SUB | SWAP
                 | ES [Command] deriving(Show, Eq)
\end{code}
\begin{code}
  data PF = POSTFIX deriving (Show, Eq)
\end{code}
\begin{code}
  type Program = (PF, Int, [Command])
\end{code}
\begin{code}
  type Stack = [Command]
\end{code}


Un ejemplo de un programa en Postfix:
\begin{code}
  (POSTFIX, 0, [ES[I 0, SWAP, SUB], I 7, SWAP, EXEC]) []
\end{code}

Las siguientes son funciones para la implementación de Postfix, algunas son recursivas.

\begin{code}
  -- | arithOperation. Funcion que realiza las operaciones de los comandos 
  --   aritmeticos. (add, div, eq, gt, lt, mul, rem, sub)
  arithOperation :: Command -> Command -> Command -> Command

  -- | stackOperation. Funcion que realiza las operaciones de los comandos 
  --   que alteran la pila de valores.(ADD, DIV, Eq, Gt, Lt, MUL, REM, SUB).
  stackOperation :: Stack -> Command -> Stack

  -- | execOperation. Funcion que devuelve la lista de comandos y
  --   la pila resultante de realizar la llamada a la operacion con exec.
  execOperation :: [Command] -> Stack -> ( [ Command ] , Stack )

  -- | validProgram. Funcion que determina si la pila de valores que
  --   se desea ejecutar con un programa es valida.
  validProgram :: Program -> Stack -> Bool

  -- | executeCommands. Funcion que dada una lista de comandos y
  --   una pila de valores obtiene la pila de valores resultant's después ejecutar
  --   todos los comandos.
  executeCommands :: [Command] -> Stack -> Stack

  -- | executeProgram. Función que ejecuta cualquier programa en Postfix.
  executeProgram :: Program -> Stack -> [Command]

\end{code}

\section{Entrada y ejecución}

El programa es interpretado por \texttt{GCHI} de la siguiente forma
\begin{lstlisting}[]
  ~:ghci Practica1.hs
\end{lstlisting}

\subsection{Postfix}
Ya en el programa, los siguientes son ejemplos de la ejecución de las funciones
para la implementación de \texttt{Postfix}.
\begin{lstlisting}[language=Haskell]

  *Practica1> arithOperation (I 1) (I 2) ADD
  I 3

  *Practica1> arithOperation (I 8) (I 0) DIV
  *** Exception: Division entre 0.

  *Practica1> stackOperation [I 1, I 5] SWAP
  [I 5, I 1]

  *Practica1> stackOperation [I 1, I 5] (ES [I 3, ADD, SWAP, I 2])
  [ES [I 3, ADD, SWAP, I 2], I 1, I 5]

  *Practica1> execOperation [ADD] [ES [I 1, ADD], I 2, I 3]
  ([I 1, ADD, ADD], [I 2, I 3])

  *Practica1> execOperation [MUL] [ES [I 1, ADD], I 2, I 3]
  ([I 1, ADD, MUL], [I 2, I 3])

  *Practica1> validProgram (POSTFIX, 0, [I 1, I 2, ADD]) []
  True

  *Practica1> validProgram (POSTFIX, 2, [ADD]) [I 3, ES[I 1, I 2, SUB]]
  False

  *Practica1> executeCommands seq1 []
  [I 3]

  *Practica1> executeCommands seq2 [I 7]
  [I 14]

  *Practica1> executeProgram (POSTFIX, 0, prg1) []
  [I (-7)]

  *Practica1> executeProgram (POSTFIX, 1, prg2) [I 4, I 5]
  *** Exception: No es un programa válido.

\end{lstlisting}
\section{Conclusiones}

Estuvo bien, el problema era ir a leer las notas e intentar entender como
funcionaban algunas cosas, por ejemplo en \texttt{arithOperation} se supone que
deberíamos restar el segundo elemento de la pila, menos el primero, pero para
que los ejemplos del PDF pasen los cambiamos.\\
Otra cosa interesante fue todos los auxiliares que tuvimos que hacer para por
ejemplo verificar en \texttt{stackOperation} pues diferentes comandos usan
diferentes tipos de pilas, por ejemplo no podíamos hacer \texttt{SWAP} de una lista con
menos de dos elementos.


\begin{thebibliography}{9}

\bibitem{lamport94}
  lp191n02.pdf, Archivero, curso de Lenguajes de Programacion 2019-1.
\end{thebibliography}

\end{document}