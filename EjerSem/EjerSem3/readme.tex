% Lenguajes de Programación 2019-1
% Plantilla para reportes de laboratorio.

\documentclass[spanish,12pt,letterpaper]{article}

\usepackage[spanish]{babel}
\usepackage[utf8]{inputenc}
\usepackage{authblk}
\usepackage{listings}
\usepackage{dsfont}
\lstnewenvironment{code}
{\lstset{}%
  \csname lst@SetFirstLabel\endcsname}
{\csname lst@SaveFirstLabel\endcsname}
\lstset{
  basicstyle=\small\ttfamily,
  flexiblecolumns=false,
  basewidth={0.5em,0.45em},
  literate={+}{{$+$}}1 {/}{{$/$}}1 {*}{{$*$}}1 {=}{{$=$}}1
  {>}{{$>$}}1 {<}{{$<$}}1 {\\}{{$\lambda$}}1
  {\\\\}{{\char`\\\char`\\}}1
  {->}{{$\rightarrow$}}2 {>=}{{$\geq$}}2 {<-}{{$\leftarrow$}}2
  {<=}{{$\leq$}}2 {=>}{{$\Rightarrow$}}2 
  {\ .}{{$\circ$}}2 {\ .\ }{{$\circ$}}2
  {>>}{{>>}}2 {>>=}{{>>=}}2
  {|}{{$\mid$}}1               
}

\title{Ejercicio Semanal 3}
\author{Emiliano Galeana Araujo}
\affil{Facultad de Ciencias, UNAM}
\date{Fecha de entrega: 08 de Noviembre de 2018}

\begin{document}

\maketitle

\section{Descripción del programa}
En este semanal implementamos un lenguaje llamado miniC, el cual es un lenguaje
imperativo que tiene como núcleo el lenguaje de expresiones aritmetico booleanas.
\subsection{MiniC}
Recordemos el lenguaje de EAB y agregamos el comportamiento de nuevos
contrusctores para miniC.
\begin{code}
  data Exp = V identifier | I Int | B Bool
    | Add Exp Exp | Mul Exp Exp | Succ Exp | Pred Exp 
    | And Exp Exp | Or Exp Exp  | Not Exp
    | Lt Exp Exp  | Gt Exp Exp  | Eq Exp Exp
    | If Exp Exp Exp
    | Let Identifier Exp Exp
    | L Int
    | Alloc Exp
    | Deref Exp
    | Assig Exp Exp
    | Void
    | Seq
    | While Exp Exp
\end{code}
Un ejemplo trivial de un programa en miniC se puede ver como sigue:
\begin{code}
  While (B True) Void
\end{code}

Agregamos nuevos tipos para poder representar la memoria y direcciones.

\begin{code}
   type Addr = Exp

   type Value = Exp

   type Cell = (Addr, Value)

   type Mem = [Cell]
\end{code}

Las siguientes son funciones que implementamos para poder representar el
lenguaje miniC.

\begin{code}
   -- | domain. Dada una memoria, obtiene todas sus direcciones.
   domain :: Mem -> [Int]

   -- | newL. Dada una memoria genera una nueva direccion.
   newL :: Mem -> Addr

   -- | accessM. Dada una direccion de memoria, devuelve el valor contenido en
   -- |          la celda con tal direccion.
   accessM :: Addr -> Mem -> Maybe Value

   -- | updateM. Dada una celda de memoria, actualiza el valor.
   updateM :: Cell -> Mem -> Mem

   -- | frVars. Extiende esta funcion del lenguaje EAB con las nuevas funciones.
   frVars :: Exp -> [Identifier]
  
   -- | subst. Extiende esta funcion del lenguaje EAB con las nuevas funciones.
   subst :: Exp -> Substitution -> Exp

   -- | eval1. Extiende esta funcion del lenguaje EAB con las nuevas funciones.
   eval1 :: (Mem, Exp) -> (Mem, Exp)

   -- | evals. Extiende esta funcion del lenguaje EAB con las nuevas funciones.
   evals :: (Mem, Exp) -> (Mem, Exp)

   -- | eval. Extiende esta funcion del lenguaje EAB con las nuevas funciones.
   eval :: Exp -> Exp

\end{code}

\section{Entrada y ejecución}

El programa es interpretado por \texttt{GCHI} de la siguiente forma
\begin{lstlisting}[]
  ~:ghci EjerSem03.hs
\end{lstlisting}

\subsection{Funciones}
Ya en el programa, los siguientes son ejemplos de ejecuciones de las funciones
antes mencionadas.

\begin{lstlisting}[language=Haskell]

  *EjerSem03> domain []
  []
  *EjerSem03> domain [(L 0, B False), (L 2, I 9)]
  [0,2]
  *EjerSem03> newL []
  []
  *EjerSem03> newL [(L 0, B False),(L 2, I 9)]
  L 1
  *EjerSem03> accessM (L 3) []
  Nothing
  *EjerSem03> accessM (L 2) [(L 0, I 21), (L 1, Void), (L 2, I 12)]
  Just (I 12)
  *EjerSem03>  
  *EjerSem03> updateM (L 3, B False) []
  ***Exception: Memory addres does not exit.
  *EjerSem03> updateM (L 0, I 22) [(L 0, I 21),(L 1, Void),(L 2, I 12)]
  [(L 0, I 22),(L 1, Void),(L 2, I 12)]
  *EjerSem03> frVars (Assig (L 2) (Add (I 0) (V ``z''')))
  [``z''']
  *EjerSem03> subst (Assig (L 2) (Add (I 0) (V ``z'''))) (``z''', B False)
  (Assig (L 2) (Add (I 0) (B False)))
  *EjerSem03> eval1 ([(L 0, B False)], (Add (I 1) (I 2)))
  ([(L 0, B False)], I 3)
  *EjerSem03> eval1 ([], While (B True) (Add (I 1) (I 1)))
  ([], If (B True) (Seq (Add (I 1) (I 1)) (While (B True) (Add (I 1) (I 1)))) Void)
  *EjerSem03> evals ([], (Let ``x''' (Add (I 1) (I 1)) (Eq (V ``x''') (I 0))))
  ([], B False)
  *EjerSem03> evals ([], Assig (Alloc (B False)) (Add (I 1) (I 9)))
  ([(L 0, I 10)], Void)
  *EjerSem03> eval (While (B True) Void)

  *EjerSem03> eval (Or (Eq (Add (I 0) (I 0)) (I 0)) (Eq (I 1) (I 1)))
  B True

\end{lstlisting}

\section{Conclusiones}

Fue algo sencillo, pues ya teníamos eval1, evals, eval del lenguaje EAB, sin
embargo, como había que hacer cosas con la memoria no podíamos pasarlo literal,
así que se tuvieron que hacer unas modificaciones, y todo salió bien, como eval1
tiene que regresar un error si le pasamos un terminal, se implementó evals
porque en algún punto no dejaba continuar con la ejecución.

Un problema fue que al atrapar errores del while, no caen ahí, sino que caen en
el if, eso es malo, pues un while true, no debería regresar un error, sin
embargo, regresa uno por parte del if.


\begin{thebibliography}{9}

\bibitem{lamport94}
  Archivero, curso de Estructuras Discretas 2017-1
\end{thebibliography}

\end{document}