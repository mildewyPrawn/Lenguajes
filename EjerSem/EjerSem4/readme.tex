% Lenguajes de Programación 2019-1
% Plantilla para reportes de laboratorio.

\documentclass[spanish,12pt,letterpaper]{article}

\usepackage[spanish]{babel}
\usepackage[utf8]{inputenc}
\usepackage{authblk}
\usepackage{listings}
\usepackage{dsfont}
\lstnewenvironment{code}
{\lstset{}%
  \csname lst@SetFirstLabel\endcsname}
{\csname lst@SaveFirstLabel\endcsname}
\lstset{
  basicstyle=\small\ttfamily,
  flexiblecolumns=false,
  basewidth={0.5em,0.45em},
  literate={+}{{$+$}}1 {/}{{$/$}}1 {*}{{$*$}}1 {=}{{$=$}}1
  {>}{{$>$}}1 {<}{{$<$}}1 {\\}{{$\lambda$}}1
  {\\\\}{{\char`\\\char`\\}}1
  {->}{{$\rightarrow$}}2 {>=}{{$\geq$}}2 {<-}{{$\leftarrow$}}2
  {<=}{{$\leq$}}2 {=>}{{$\Rightarrow$}}2 
  {\ .}{{$\circ$}}2 {\ .\ }{{$\circ$}}2
  {>>}{{>>}}2 {>>=}{{>>=}}2
  {|}{{$\mid$}}1               
}

\title{Ejercicio Semanal 4}
\author{Emiliano Galeana Araujo}
\affil{Facultad de Ciencias, UNAM}
\date{Fecha de entrega: 15 de Noviembre de 2018}

\begin{document}

\maketitle

\section{Descripción del programa}
En este semanal implementamos máquinas K para evaluar expresiones EAB.
\subsection{máquina K}
Recordemos el lenguaje de EAB y agregamos el tipo Error. También agregamos
nuevos datos, para poder regresentar las evaluaciones de los marcos, al cual
llamamos Frame.
\begin{code}
  data Exp = V identifier | I Int | B Bool
    | Add Exp Exp | Mul Exp Exp | Succ Exp | Pred Exp 
    | And Exp Exp | Or Exp Exp  | Not Exp
    | Lt Exp Exp  | Gt Exp Exp  | Eq Exp Exp
    | If Exp Exp Exp
    | Let Identifier Exp Exp
    | Error

  data Frame = AddL Pending Expr
           | AddR Expr Pending
           | MulL Pending Expr
           | MulR Expr Pending
           | SuccF Pending
           | PredF Pending
           | AndL Pending Expr
           | AndR Expr Pending
           | OrL Pending Expr
           | OrR Expr Pending
           | NotF Pending
           | LtL Pending Expr
           | LtR Expr Pending
           | GtL Pending Expr
           | GtR Expr Pending
           | EqL Pending Expr
           | EqR Expr Pending
           | IfF Pending Expr Expr
           | LetF Identifier Pending Expr
\end{code}
Un ejemplo de un programa en una máquina K es el siguiente, el cuál regresa un
error.
\begin{code}
  eval (Not (I 3))
\end{code}

Agregamos un nuevo tipo para poder representar un marco, y su evaluación.

\begin{code}
   type Pending ()
\end{code}

Las siguientes son funciones que implementamos para poder representar las
máquinas K.

\begin{code}
   -- | frVars. Obtiene el conjunto de variables libres de una expresion.
   frVars :: Exp -> [Identifier]
  
   -- | subst. Realiza la substitucion de una expresion de EAB.
   subst :: Exp -> Substitution -> Exp

   -- | eval1. Recibe un estado de la maquina K, y devuelve un paso de la
   -- |        transicion.
   eval1 :: (Mem, Exp) -> (Mem, Exp)

   -- | evals. Recibe un estado de la maquina K y devuelve un estado derivado de
   -- |        evaluar varias veces hasta obtener la pila vacia.
   evals :: (Mem, Exp) -> (Mem, Exp)

  -- | eval. Recibe una expresion EAB, la evalua con la maquina K, y devuelve un
  -- |       valor, iniciando con la pila vacia esta devuelve un valor a la pila.
   eval :: Exp -> Exp

\end{code}

\section{Entrada y ejecución}

El programa es interpretado por \texttt{GCHI} de la siguiente forma
\begin{lstlisting}[]
  ~:ghci EjerSem04.hs
\end{lstlisting}

\subsection{Funciones}
Ya en el programa, los siguientes son ejemplos de ejecuciones de las funciones
antes mencionadas.

\begin{lstlisting}[language=Haskell]
  *EjerSem04> frVars (Add (V ``x''') (I 5))
  [``x''']
  *EjerSem04> frVars (Let ``x''' (I 1) (V ``x'''))
  []
  *EjerSem04> subst (Add (V ``x''') (I 5)) (``x''', I 10)
  Add (I 10) (I 5)
  *EjerSem04> subst (Let ``x''' (I 1) (V ``x''')) (``y''', Add (V ``x''') (I 5))
  ***Exceotion: Could not apply the substitucion.
  *EjerSem04> eval1 (E ([], Add (I 1) (I 2)))
  *EjerSem04> eval1 (R ([OrR () (B True), AndL () (B False)], B False))
  *EjerSem04> evals (E ([], Let ``x''' (I 2) 
                                       (Mul (Add (I 1) (V ``x''')) (V ``x'''))))
  R ([], I 6)
  *EjerSem04> evals (E ([], Let ``x''' (B True) 
                                       (If (V ``x''') (V ``x''') (B False))))
  R ([], B True)
  *EjerSem04> evals (E ([], Add (B True) (I 1)))
  R ([], Error)
  *EjerSem04> eval (Let ``x''' (I 1)
                               (If (Gt (V ``x''') (I 0)) 
                                   (Eq (V ``x''') (I 0)) (B True)))
  B False
  *EjerSem04> eval (Not (I 3))
  *** Exception: Error.

\end{lstlisting}

\section{Conclusiones}
Fue sencillo realizar el semanal después de haber hecho los anteriores y ver
como se comportaban eval1, evals, eval, frVars, subst, la parte complicada fue
teníamos que tener algo de la forma R R, E E, E R, R E, y en las expresiones
binarias creía que me faltaba uno, pero no iba, así que una vez entendido eso,
la implementación fue sencilla, aparte de que el evals y eval fueron más sencillos.
\begin{thebibliography}{9}

\bibitem{lamport94}
  Archivero, curso de Estructuras Discretas 2017-1
\end{thebibliography}

\end{document}